% 基础包
\usepackage[utf8]{inputenc}
\usepackage{xeCJK}      % 中文支持
\usepackage{graphicx}   % 图片支持
\usepackage{geometry}   % 页面布局
\usepackage{titlesec}   % 标题控制
\usepackage{longtable}  % 长表格支持
\usepackage{booktabs}   % 表格美化
\usepackage{listings}   % 代码块支持
\usepackage{xcolor}     % 颜色支持
\usepackage{framed}     % 框架支持
\usepackage{fancyvrb}   % 代码高亮支持
\usepackage{fvextra}    % 扩展的代码支持

% hyperref 包需要放在大多数包的后面
\usepackage{hyperref}   % URL 支持
\hypersetup{
    colorlinks=true,
    linkcolor=blue,
    urlcolor=blue,
    bookmarksnumbered=false,     % 书签中不显示编号
    bookmarksopen=true,          % 展开书签
    pdfstartview=FitH            % PDF 打开时适应页面宽度
}
% 关闭章节编号
\setcounter{secnumdepth}{-2}
% 重新定义章节格式,保持超链接
\titleformat{\section}
  {\normalfont\Large\bfseries}   % 格式
  {}                             % 标签
  {0pt}                          % 分隔符
  {\phantomsection}              % 前置代码:添加 PDF 标签
% 设置新章节另起一页,同时确保正确的页码引用
\newcommand{\sectionbreak}{%
  \clearpage
  \phantomsection  % 为下一个章节创建超链接标记点
}

% 定义代码高亮颜色
\definecolor{shadecolor}{RGB}{248,248,248}
\definecolor{comment}{RGB}{128,128,128}
\definecolor{keyword}{RGB}{0,112,192}
\definecolor{string}{RGB}{0,136,0}
\definecolor{identifier}{RGB}{0,0,0}
\definecolor{number}{RGB}{128,0,128}

% 定义代码高亮环境
\newenvironment{Shaded}
    {\begin{snugshade}\small\ttfamily}
    {\end{snugshade}}

\newenvironment{Highlighting}
    {\begin{Verbatim}[commandchars=\\\{\},fontsize=\small,breaklines=true,breakanywhere=true]}
    {\end{Verbatim}}

% 定义所有可能的代码高亮 token
\newcommand{\KeywordTok}[1]{\textcolor{keyword}{\textbf{#1}}}
\newcommand{\DataTypeTok}[1]{\textcolor{keyword}{#1}}
\newcommand{\DecValTok}[1]{\textcolor{number}{#1}}
\newcommand{\BaseNTok}[1]{\textcolor{number}{#1}}
\newcommand{\FloatTok}[1]{\textcolor{number}{#1}}
\newcommand{\CharTok}[1]{\textcolor{string}{#1}}
\newcommand{\StringTok}[1]{\textcolor{string}{#1}}
\newcommand{\CommentTok}[1]{\textcolor{comment}{\textit{#1}}}
\newcommand{\OtherTok}[1]{\textcolor{identifier}{#1}}
\newcommand{\AlertTok}[1]{\textcolor{red}{#1}}
\newcommand{\FunctionTok}[1]{\textcolor{identifier}{#1}}
\newcommand{\RegionMarkerTok}[1]{#1}
\newcommand{\ErrorTok}[1]{\textbf{\textcolor{red}{#1}}}
\newcommand{\NormalTok}[1]{#1}
\newcommand{\OperatorTok}[1]{\textcolor{keyword}{#1}}
\newcommand{\BuiltInTok}[1]{\textcolor{keyword}{#1}}
\newcommand{\ExtensionTok}[1]{#1}
\newcommand{\PreprocessorTok}[1]{\textcolor{keyword}{#1}}
\newcommand{\AttributeTok}[1]{\textcolor{keyword}{#1}}
\newcommand{\RegionTok}[1]{#1}
\newcommand{\InformationTok}[1]{\textcolor{keyword}{#1}}
\newcommand{\WarningTok}[1]{\textcolor{red}{#1}}
\newcommand{\AnnotationTok}[1]{\textcolor{comment}{#1}}
\newcommand{\CommentVarTok}[1]{\textcolor{comment}{\textit{#1}}}
\newcommand{\VariableTok}[1]{\textcolor{identifier}{#1}}
\newcommand{\ControlFlowTok}[1]{\textbf{#1}}
\newcommand{\SpecialCharTok}[1]{\textcolor{string}{#1}}
\newcommand{\SpecialStringTok}[1]{\textcolor{string}{#1}}
\newcommand{\VerbatimStringTok}[1]{\textcolor{string}{#1}}
\newcommand{\ConstantTok}[1]{\textcolor{number}{#1}}
\newcommand{\ImportTok}[1]{\textcolor{keyword}{#1}}
\newcommand{\DocumentationTok}[1]{\textcolor{comment}{\textit{#1}}}
\newcommand{\SectionTok}[1]{\textbf{#1}}

% 图片设置
\DeclareGraphicsExtensions{.pdf,.png,.jpg,.jpeg}
\graphicspath{{./}{./images/}}
% 修改图片配置,使其居中显示
\makeatletter
\g@addto@macro\@floatboxreset{%
    \centering  % 添加居中命令
    \linewidth=0.9\textwidth
}
\makeatother
\let\oldincludegraphics\includegraphics
\renewcommand{\includegraphics}[2][]{\oldincludegraphics[width=\linewidth,keepaspectratio,#1]{#2}}

% 页面设置
\geometry{
    a4paper,
    margin=2.5cm
}

% 中文字体设置
\setCJKmainfont{PingFang SC}

% 超链接设置
\hypersetup{
    colorlinks=true,
    linkcolor=blue,
    urlcolor=blue
}

% 定义基本命令
\providecommand{\tightlist}{%
  \setlength{\itemsep}{0pt}\setlength{\parskip}{0pt}}

% 定义 pandoc 相关命令
\providecommand{\pandocbounded}[1]{#1}

% 代码块设置
\lstset{
    breaklines=true,
    basicstyle=\ttfamily\small,
    frame=single,
    numbers=left,
    numberstyle=\tiny,
    keywordstyle=\color{keyword},
    commentstyle=\color{comment},
    stringstyle=\color{string},
    showstringspaces=false,
    backgroundcolor=\color{shadecolor},
    breakatwhitespace=false,
    breakindent=0pt,
    keepspaces=true,
    columns=flexible,
    escapeinside={(*@}{@*)},
    linewidth=\textwidth,     % 设置代码块宽度为页面宽度
    xleftmargin=0pt,         % 移除左边距
    xrightmargin=0pt,        % 移除右边距
    belowskip=\medskipamount, % 设置代码块下方间距
    aboveskip=\medskipamount  % 设置代码块上方间距
}

% 图片错误处理
\usepackage{silence}
\WarningFilter{latex}{File}
\WarningFilter{latex}{Cannot determine size}

% 设置代码块换行
\DefineVerbatimEnvironment{Highlighting}{Verbatim}{
    commandchars=\\\{\},
    breaklines=true,
    breakanywhere=true,
    xleftmargin=0pt,         % 移除左边距
    xrightmargin=0pt,        % 移除右边距
    baselinestretch=1.2,     % 设置行间距
    fontsize=\small          % 设置字体大小
}

% 重新定义 Shaded 环境
\renewenvironment{Shaded}{
    \begin{snugshade}
    \begin{adjustwidth}{0pt}{0pt} % 移除左右边距
}{
    \end{adjustwidth}
    \end{snugshade}
}

% 添加 adjustwidth 环境支持
\usepackage{changepage}